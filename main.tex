\documentclass[a4paper,12pt]{article}

\usepackage[portuguese]{babel}		% Hifenização português
\usepackage[utf8]{inputenc}			% Codificação
\usepackage{url}					% Reconhecimento de links
\usepackage[top=3cm,
			bottom=1.5cm,
			left=2.79cm,
			right=1.83cm]{geometry}	% Tamanho das bordas
\usepackage{indentfirst}			% Indentação da primeira linha
\usepackage{graphicx}				% Inserir imagens
\usepackage{fancyhdr}				% Cabeçalhos e rodapes personalizados

\pagestyle{fancy}
\fancyhf{}
\chead{AQUI VAI A IMAGEM DE CABEÇALHO}
\cfoot{AQUI COLOCAMOS A IMAGEM DE RODAPÉ}

%%%%% Texto de exemplo %%%%%
\usepackage{lipsum}					% Texto de exemplo do tipo lorem ipsum

\title{
	Um título\\
	E um subtítulo
}
\author{
	Autor 1\footnote{Dados do autor 1},
	Autor 2\footnote{Dados do autor 2},
	Autor 3\footnote{Dados do autor 3},
	Autor 4\footnote{Dados do autor 4},
	Autor 5\footnote{Dados do autor 5},
}



%%%%%%%%%%%%%%%%%%%%%%%%%%%%%%%%%%%%%%%%%%%%%%%%%%%
%%%%%%%%%%%%%%% Início do documento %%%%%%%%%%%%%%%
%%%%%%%%%%%%%%%%%%%%%%%%%%%%%%%%%%%%%%%%%%%%%%%%%%%
\begin{document}
\maketitle
\thispagestyle{fancy}

\section*{\center Resumo}
\lipsum[1-2]

Palavras-chaves: palavra1, palavra2, palavra3.

\section*{\center Introdução e Justificativa}
\lipsum[1-10]

\section*{\center Metodologia}
\lipsum[1-10]

\section*{\center Resultados e Discussões}
\lipsum[1-10]

\section*{\center Considerações Finais}
\lipsum[1-10]

\section*{\center Agradecimentos}
\lipsum[1]

\section*{\center Referências}
\lipsum[1-3]

\end{document}
